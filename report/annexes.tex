\chapter{Codes sources utilisés}\label{ann:src}
\renewcommand{\leftmark}{ANNEXE \thechapter.~~Codes sources utilisés}
\label{annexe1}

Afin de comprendre comment fonctionnent certains logiciels étudiés dans
l'étude de faisabilité, certains codes sources disponibles en ligne
ont été utilisés. Ceux-ci étant disponibles sous licences libres (GPL et MIT),
ils sont non seulement disponibles en ligne mais peuvent aussi être
réutilisés librement.

Bien que ces codes ne seront pas réutilisés pour des raisons évidentes
(utilisation d'un environnement de développement différent), certains
principes ont été compris grâce à ceux-ci. De même les principes de base
utilisé dans le projet s'inspirent largement de ces logiciels, c'est pour
cette raison qu'une annexe leur est consacrée.

\section*{ClipboardMultiSharer}
Le code source Java de l'application sous licence GPL
\footnote{\url{http://sourceforge.net/projects/clipboardmshare/develop}}
a été parcouru afin de comprendre comment fonctionnait
le logiciel, entre autre afin de voir comment il était possible d'accéder
au presse-papier en Java.
La révision consultée était la 16 datant du 10 juillet 2010.

\section*{Clipboard Share}
Bien que disponible sous licence MIT
\footnote{\url{http://clipboardshare.codeplex.com/SourceControl/list/changesets}}, n'ayant pas de connaissances en C\# et en MS .NET, le code de ce logiciel
n'a été ni utilisé dans un but précis, ni lu.

\section*{The Network Clipboard}
Le code source sous licence GPL
\footnote{\url{http://sourceforge.net/projects/netclipboard/develop}}
a été consulté dans le but d'examiner le protocole utilisé. La révision
consultée était la 88.

\section*{Remote Clip}
Le code source sous licence GPL est disponible avec la dernière version
du logiciel
\footnote{\url{http://www.cs.cmu.edu/~rcm/RemoteClip/RemoteClip-3.1.zip}}.
Celui-ci a été consulté dans le but d'étudier le protocole réseau utilisé.

\section*{Gestionnaires de presse-papier}
Le code source de plusieurs gestionnaires de presse-papier ont permis d'aider
à la compréhension du fonctionnement du presse-papier via l'utilisation de
librairies annexes telles que GTK et Qt.
Ces logiciels sont Glipper 2.1\footnote{Version 2.1:
\url{https://launchpad.net/glipper}},
Klipper\footnote{Version présente dans KDE 4.6 (\url{http://www.kde.org})} et
parcellite \footnote{version 1.0.1: \url{http://parcellite.sourceforge.net}}.

\chapter{Principe KISS}\label{ann:kiss}
\renewcommand{\leftmark}{ANNEXE \thechapter.~~Principe KISS}
\label{annexe2}
Le principe \emph{KISS} (Keep It Simple, Stupid!) \cite{wiki:kiss} est un
principe prônant la simplicité, tout particulièrement dans le monde de la
conception.
Il est utilisé dans le développement logiciel dans le but d'exprimer le fait
que la conception doit être simple et éviter toute complexité inutile.

Celui-ci est à la base même de la philosophie Unix, en effet celle-ci prône
l'utilisation de petits utilitaires simples ayant chacun une fonctionnalité
bien précise.
Douglas McIlroy, inventeur du pipe UNIX, a dit\cite{quartercentury-unix}:
\begin{quote}
  This is the Unix philosophy: Write programs that do one thing and do
  it well. Write programs to work together. Write programs to handle
  text streams, because that is a universal interface.
\end{quote}
Le projet a donc été développé en utilisant une approche orientée
composant\cite{wiki:poc}, en fournissant de petits
logiciels communiquant au travers de sockets de manière textuelle.

Plusieurs auteurs
\cite{Brooks1995, Raymond2001} prônent cette philosophie sans pour autant
y faire référence explicitement. \emph{e.g.} Brooks explique par exemple
que le design est d'une importance capitale car il permet de comprendre
plus facilement le code source, Raymond dit qu'il faut concevoir les logiciels
de manière à ce qu'ils suivent la philosophie d'Unix. Il dit aussi que
la perfection en conception n'est pas atteinte lorsque l'on n'a plus rien à
ajouter mais plutôt lorsque l'on n'a plus rien à retirer.

Filip Hanik, un ingénieur logiciel membre de l'\emph{Apache Foundation} parle
aussi du principe KISS et donne des conseils pour appliquer ces principes en
Java (mais ceux-ci sont applicables dans tous les langages de programmation)
\cite{fhanikKISS}.
