\chapter*{Introduction}
\addcontentsline{toc}{chapter}{Introduction}
\renewcommand{\leftmark}{INTRODUCTION}

Lorsque l'on travaille sur un ordinateur, il est souvent plus agréable
de travailler en utilisant plusieurs écrans. Cela permet par exemple
d'afficher des informations sur un écran tout en prenant note sur un autre.
De plus lorsqu'un utilisateur dispose de deux ordinateurs, par exemple
un PC fixe et un portable, il en arrive à travailler en utilisant plusieurs
écrans.

Seulement une différence majeure existe entre le fait de travailler
avec plusieurs écrans sur un ordinateur et le fait de travailler avec
plusieurs ordinateurs ayant chacun leur écran. Dans le premier cas il est
aisé de faire transiter de l'information d'un écran à un autre, ceux-ci
étant des périphériques reliés au même ordinateur. Dans le second cas
cela est impossible sans passer par un moyen de communication tel que l'e-mail
ou un support externe tel qu'une clé USB. Or cela est fastidieux
s'il faut transmette peu d'information fréquemment.

Le but visé ici est donc de résoudre ce problème grâce
à un système de copier/coller entre plusieurs plateformes \emph{i.e.} entre
plusieurs machines connectées sur le même réseau.
Pour cela un premier chapitre permettra de présenter le problème, d'énoncer
les objectifs et les contraintes du projet. Ensuite un second chapitre
permettra d'analyser les solutions existantes qui permettent de résoudre
le problème. Ces solutions seront présentées, leurs avantages et inconvénients
seront donnés et elles seront finalement comparées.

Un troisième chapitre
permettra de définir une architecture réseau, ainsi qu'un protocole de
communication, à utiliser pour implémenter un ensemble de logiciels
permettant de résoudre le problème de copier/coller multi-plateformes.
Suivra un quatrième chapitre présentant l'implémentantion de cette solution,
décrivant et justifiant le choix des les outils utilisés pour la mettre en
oeuvre.

Enfin une conclusion permettra de rendre compte de l'avancement du travail
réalisé et donnera les grandes lignes qui guideront l'évolution future
du logiciel nouvellement implémenté.