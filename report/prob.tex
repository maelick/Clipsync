\chapter{Problème}\label{ch:1}
\renewcommand{\leftmark}{\thechapter.~~Problème}
\section{Objectifs}
Le premier objectif de ce projet est de donner à un utilisateur la possibilité
de faire du copier/coller entre plusieurs ordinateurs allumés. Le logiciel
créé doit être une solution, simple et légère, à installer sur chaque
poste client. C'est-à-dire qu'il doit être facilement configurable, afin
de pouvoir rapidement fournir le service désiré, et il doit avoir
une charge minimale sur le système.
Il doit permettre de faire facilement l'opération de copier/coller
sans avoir à passer par un échange de mails ou de fichiers.
Il doit aussi viser le monde Unix en premier lieu, mais doit être adaptable
à d'autres systèmes d'exploitation.

Afin de fixer certains termes plusieurs définitions sont nécessaires.
\begin{defi}
  Un \emph{presse-papier} est une zone mémoire dans laquelle une ou
  plusieurs données sont stockées. Celles-ci peuvent être de différents types,
  \emph{e.g.} du texte, une image, un fichier.
\end{defi}
\begin{rem}
  Bien qu'habituellement un presse-papier ne permet de stocker qu'une seule
  donnée à la fois, il existe un grand nombre de logiciels permettant
  de gérer un presse-papier contenant un ensemble de données, ce qui permet
  d'avoir ainsi un \emph{historique} de ce qui a été copié dedans.
  \emph{e.g.} GNU Emacs \cite{emacs}
  permet de cycler sur cet historique en revenant au début de celui-ci
  lorsqu'il a été entièrement parcouru. L'environnement de bureau Xfce inclut
  un plugin permettant de gérer l'historique du presse-papier de X Window
  \cite{xfce-clipman}
\end{rem}
\begin{defi}
  \emph{Copier} est l'action d'écrire une donnée dans le presse-papier.
\end{defi}
\begin{defi}
  \emph{Coller} est l'action de récupérer la donnée présente dans le
  presse-papier.
\end{defi}
\begin{defi}
  \emph{Copier/coller} résume le concept permettant de copier
  quelque chose dans le presse-papier et le coller ensuite.
\end{defi}
\begin{defi}
  Copier/coller \emph{multi-plateformes} décrit la possibilité de faire
  du copier/coller entre deux ordinateurs connectés en réseau. L'abréviation
  \emph{CCMP} sera parfois utilisée dans la suite de ce rapport
  \footnote{Un protocole de chiffrement de la norme IEEE 802.11i est
  aussi abrégé CCMP, cependant celui-ci n'ayant aucun lien avec le sujet
  du projet, l'utilisation de cette abréviation ne créera pas d'ambiguïté.}.
\end{defi}

\section{Contraintes}
La priorité est de fournir un système léger. Celui-ci doit être
facilement installable et configurable, discret et l'\emph{overhead}
sur le système doit être minimisé. Il ne doit donc pas reposer
sur un autre système plus lourd comme le partage de fichiers.
Dans un premier temps, il ne doit gérer que le texte mais il doit être
extensible. Ceci afin de permettre facilement l'ajout du support pour
d'autres formats de données comme les images.
Une autre contrainte importante à prendre en considération
est l'aspect réseau. Celui-ci introduit de potentiels problèmes de sécurité.
Il faudra donc trouver un moyen de réduire ceux-ci \emph{e.g.} en chiffrant
la connexion réseau et en obligeant l'utilisateur à s'authentifier
afin d'utiliser le logiciel. De même, le logiciel devant tourner sur
plusieurs systèmes d'exploitations différents dont Linux, il est
plus que préférable que le logiciel soit libre de droits et se base
sur des protocoles et technologies ouverts.
