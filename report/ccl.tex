\chapter*{Conclusion}
\addcontentsline{toc}{chapter}{Conclusion}
\renewcommand{\leftmark}{CONCLUSION}
Finalement, après avoir examiné le problème, les solutions existantes
pour le résoudre se sont avérées pour la plupart inefficaces. Une solution
reprenant un grand nombre d'idées de The Network Clipboard et de Remote
Clip a été imaginée pour ensuite être implémentée. Cette solution
comprend un mécanisme de broadcast pour la découverte de pairs et un mécanisme
d'authentification permettant de s'assurer que le presse-papier ne pourra ni
être altéré, ni être lu par un utilisateur du réseau local étrangé au groupe
d'ordinateurs entre lesquels l'utilisateur souhaite synchroniser le
presse-papier.

Clipsync, le logiciel produit, a ainsi pu être publié comme logiciel libre sur
Launchpad dans le but de faciliter sa distribution ainsi que son évolution.
Pour plannifier cette évolution, les fonctionnalités manquantes ont été mises
en avant.
Le passage à l'IPv6 étant un problème d'actualité urgent, l'effort
sera avant tout concentré sur son support dans Clipsync ainsi que sur la
découverte et la correction debugs. Ceci permettra de stabiliser le logiciel
avant l'ajout de nouvelles fonctionnalités.

Ensuite l'ajout de nouvelles fonctionnalités pour l'utilisateur telles que
l'implémentation du copier/coller d'images et la compression du presse-papier,
seront la seconde priorité des développements futurs du logiciel.
A plus long terme, l'utilisation de mécanismes d'auto-configuration
standardisés tels qu'Avahi sera utilisée pour remplacer le mécanisme
de broadcast.