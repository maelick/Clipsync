\chapter*{Conclusion}
\addcontentsline{toc}{chapter}{Conclusion}
\renewcommand{\leftmark}{CONCLUSION}
Finalement, après avoir examiné le problème, les solutions existantes
pour le résoudre se sont avérées pour la plupart inefficaces. Certaines
ont pourtant de bonnes idées qui ont été reprises afin d'imaginer une
architecture et une ébauche de protocole permettant à des pairs de synchroniser
leur presse-papier sur un réseau. Grâce à cela, il est désormais possible
d'implémenter ce programme afin de fournir une solution efficace.

Pour cela le langage C++ sera utilisé, celui-ci permettant d'avoir un programme
compilé optimisé pour un système précis, il permet de mieux satisfaire
la contrainte de solutions légères qu'avec un langage interprété par une
machine virtuelle. De même le C++ offre l'avantage de donner l'accès
directement à la Xlib et de communiquer facilement avec le serveur
X Window.

La librairie Qt, célèbre pour être utilisée dans l'environnement de bureau
KDE, sera aussi utilisée. Celle-ci permet cependant de faire bien plus que des
programmes fenêtrés. Elle contient entre autre un ensemble de classes
permettant d'établir des connexions TCP/TLS de manière indépendante du
système d'exploitation. Par manque d'expérience avec cette librairie,
il n'est cependant pas possible de dire ce qu'elle pourrait apporter en plus,
celle-ci contenant un grand nombre d'outils différents.

Enfin les prochains mois seront consacrés, tout d'abord à l'étude de
la librairie Qt (et en particulier de sa partie concernant les sockets).
Ensuite une fois ceci fait, il sera possible d'implémenter l'ensemble
des logiciels à fournir de manière incrémentale. Le client P2P sera
d'abord développé afin d'offrir un protocole de base fonctionnel.
Une fois celui-ci implémenté et testé, il sera alors possible
de passer à l'implémentation de trois types de clients locaux.

Un premier permettant une utilisation simple sur l'entrée
et la sortie standard d'un terminal Unix. Le second devra synchroniser le
presse-papier de X Window. Le dernier fournira une interface graphique
basique pour les systèmes ne supportant pas X, il permettra de copier/coller
du texte dans un champ texte et de le synchroniser sur le réseau.
Cependant si une solution est trouvée afin de fournir un client
permettant de synchroniser le presse-papier indépendemment du système
d'exploitation, les deux clients n'en formeront qu'un seul.