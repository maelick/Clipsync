\documentclass[12pt,a4paper,oneside, titlepage]{report}

\usepackage{times}
\usepackage[french,frenchb]{babel}
%\usepackage{hyperref}
\usepackage[utf8]{inputenc}
\usepackage[T1]{fontenc}
%\usepackage{amsmath}
%\usepackage{amsfonts}
%\usepackage{amscd}
%\usepackage{amstext}
%\usepackage{amssymb}
%\usepackage{bar}
\usepackage{color}
%\usepackage{mathrsfs}
\usepackage{graphicx}
%\usepackage{calligra}
\usepackage{amsthm}
%\usepackage{multirow}
%\usepackage{tabularx}
%\usepackage{layout}
%\pagestyle{headings}
\usepackage{fancyhdr}
\usepackage{array}
\usepackage{lscape}
\usepackage{rotating}
\usepackage[style=alphabetic,sortcites=true,block=space]{biblatex}
% \usepackage{pstricks}
\usepackage{pstricks}
\pagestyle{fancy}

\bibliography{biblio}

\setlength{\textheight}{630pt}
\setlength{\footskip}{30pt}
\newtheorem{defi}{D\'efinition}[section]
\newtheorem{note}{Note}[section]
\newtheorem{propriete}{Propri\'et\'e}[section]
\newtheorem{exemple}{Exemple}[section]
\newtheorem{exemple_util}{Exemple d'utilisation}
\newtheorem{corollaire}{Corollaire}[section]
\newtheorem{rem}{Remarque}[section]
\newtheorem{thm}{Th\'eor\`eme}[section]
\newtheorem{illustration}{Illustration}[section]
\newenvironment{demonstration}{\begin{proof}[\textnormal{\textbf{Preuve.}}]}{\end{proof}}
\definecolor{gris}{gray}{0.45}
\setlength{\parindent}{1cm}
\newcommand{\textcalli}[1]{{\small{\textbf{$\negmedspace$\calligra #1}}}}

% \renewcommand{\chaptermark}[1]{\markright{\thechapter\ #1}}
\renewcommand{\sectionmark}[1]{\markright{\thesection\ #1}}
\fancyhf{} % supprime les en-têtes et pieds prédéfinis
\fancyhead[R]{\thepage}% Left Even, Right Odd
\fancyhead[L]{\textsl{\leftmark}} % Left Odd
%\fancyhead[RE]{\textsl{\leftmark}} % Right Even
\renewcommand{\headrulewidth}{0pt}% filet en haut de page
\renewcommand{\footrulewidth}{0pt} % pas de filet en bas
\fancypagestyle{plain}{ % pages de tetes de chapitre
\fancyhead{} % supprime l'entete
\fancyhead[R]{\thepage}
\renewcommand{\headrulewidth}{0pt} % et le filet
}

\begin{document}
%newpage
%\thispagestyle{empty}
%\null
%\newpage
\pagenumbering{roman}
\begin{titlepage}
% \vspace*{0.95cm}
\begin{center}
\textnormal{\Large{Universit\'e de Mons}}\\[0.3em]
\textnormal{\Large{Facult\'e des Sciences}}\\[0.3em]
\textnormal{\Large{Institut d'Informatique}}\\[0.3em]
\end{center}
\vspace*{4cm}
\begin{center}
\fbox{
\begin{minipage}{14cm}
\center
\vspace*{0.5cm}\textbf{\LARGE{Copier/coller multi-plateformes}}\\
\end{minipage}
}
\end{center}
\vspace*{3cm}

\large{
\begin{center}
\begin{tabular*}{14.5cm}{@{\extracolsep{\fill}}lr}
Directeur : M\textsuperscript{r} Olivier \textsc{Delgrange} &
Projet r\'ealis\'e par\\
& Maëlick \textsc{Claes}\\[1em]
Rapporteurs : M\textsuperscript{r} Bruno \textsc{Quoitin} & \\
\hspace{28.9mm}M\textsuperscript{r} Sylvain \textsc{Degrandsart} &
\end{tabular*}
\end{center}}

\vspace*{2cm}
\begin{center}
\includegraphics[height=2cm]{umons}
% \hspace{0.5cm}
% \includegraphics[height=1.7cm]{logo_academie}
\\[1em]
Ann\'ee acad\'emique 2010-2011
\end{center}

\end{titlepage}


% \section*{Remerciements}
% \renewcommand{\leftmark}{REMERCIEMENTS}
% %\addcontentsline{toc}{chapter}{Remerciements}

% Je remercie ...\\

\newpage
\renewcommand{\leftmark}{TABLE DES MATI\`{E}RES}
\thispagestyle{fancy}
\tableofcontents

\newpage
\pagenumbering{arabic}
\chapter*{Introduction}
\addcontentsline{toc}{chapter}{Introduction}
\renewcommand{\leftmark}{INTRODUCTION}

Lorsque l'on travaille sur un ordinateur, il est souvent plus agréable
de travailler en utilisant plusieurs écrans. Cela permet par exemple
d'afficher des informations sur un écran tout en prenant note sur un autre.
De plus lorsqu'un utilisateur dispose de deux ordinateurs, par exemple
un PC fixe et un portable, il en arrive à travailler en utilisant plusieurs
écrans.

Seulement une différence majeure existe entre le fait de travailler
avec plusieurs écrans sur un ordinateur et le fait de travailler avec
plusieurs ordinateurs ayant chacun leur écran. Dans le premier cas il est
aisé de faire transiter de l'information d'un écran à un autre, ceux-ci
étant des périphériques reliés au même ordinateur. Dans le second cas
cela est impossible sans passer par un moyen de communication tel que l'e-mail
ou un support externe tel qu'une clé USB. Or cela est fastidieux
s'il faut transmette peu d'information fréquemment.

Le but visé ici est donc de résoudre ce problème grâce
à un système de copier/coller entre plusieurs plateformes \emph{i.e.} entre
plusieurs machines connectées sur le même réseau.
Pour cela un premier chapitre permettra de présenter le problème, d'énoncer
les objectifs et les contraintes du projet. Ensuite un second chapitre
permettra d'analyser les solutions existantes qui permettent de résoudre
le problème. Ces solutions seront présentées, leurs avantages et inconvénients
seront donnés et elles seront finalement comparées.

Un troisième chapitre
permettra de définir une architecture réseau, ainsi qu'un protocole de
communication, à utiliser pour implémenter un ensemble de logiciels
permettant de résoudre le problème de copier/coller multi-plateformes.
Suivra un quatrième chapitre présentant l'implémentantion de cette solution,
décrivant et justifiant le choix des outils utilisés pour la mettre en
oeuvre. Une évaluation des performances sera aussi faite et les
fonctionnalités à améliorer ou à ajouter dans le futur seront identifiées.

Enfin une conclusion permettra de rendre compte de l'avancement du travail
réalisé et donnera les grandes lignes qui guideront l'évolution future
du logiciel nouvellement implémenté.
\chapter{Problème}\label{ch:1}
\renewcommand{\leftmark}{\thechapter.~~Problème}
\section{Objectifs}
Le premier objectif de ce projet est de donner à un utilisateur la possibilité
de faire du copier/coller entre plusieurs ordinateurs allumés. Le logiciel
créé doit être une solution, simple et légère, à installer sur chaque
poste client. C'est-à-dire qu'il doit être facilement configurable, afin
de pouvoir rapidement fournir le service voulu, et il doit avoir
une charge minimale sur le système.
Il doit permettre de faire facilement l'opération de copier/coller
sans avoir à passer par un échange de mails ou de fichiers.
Il doit aussi viser le monde Unix en premier lieu, mais doit être adaptable
à d'autres systèmes d'exploitation.

Afin de fixer certains termes plusieurs définitions sont nécessaires.
\begin{defi}
  Un \emph{presse-papier} est une zone mémoire dans laquelle une ou
  plusieurs données sont stockées. Celles-ci peuvent être de différents types,
  \emph{e.g.} du texte, une image, un fichier.
\end{defi}
\begin{rem}
  Bien qu'habituellement un presse-papier ne permet de stocker qu'une seule
  donnée à la fois, il existe un grand nombre de logiciels permettant
  de gérer un presse-papier contenant un ensemble de données permettant
  d'avoir ainsi un \emph{historique} de ce qui a été copié dedans.
  \emph{e.g.} GNU Emacs \cite{emacs}
  permet de cycler sur cet historique en revenant au début de celui-ci
  lorsqu'il a été entièrement parcouru.
\end{rem}
\begin{defi}
  \emph{Copier} est l'action d'écrire une donnée dans le presse-papier.
\end{defi}
\begin{defi}
  \emph{Coller} est l'action de récupérer la donnée présente dans le
  presse-papier.
\end{defi}
\begin{defi}
  \emph{Copier/coller} résume le concept permettant de copier
  quelque chose dans le presse-papier et le coller ensuite.
\end{defi}
\begin{defi}
  Copier/coller \emph{multi-plateformes} décrit la possibilité de faire
  du copier/coller entre deux ordinateurs mis en réseau. L'abréviation
  \emph{CCMP} sera parfois utilisée dans la suite de ce rapport
  \footnote{Un protocole de chiffrement de la norme IEEE 802.11i est
  aussi abrégé CCMP, cependant celui-ci n'ayant aucun lien avec le sujet
  du projet, l'utilisation de cette abréviation ne créera pas d'ambiguïté.}.
\end{defi}

\section{Contraintes}
La priorité est de fournir un système léger. Celui-ci doit être
facilement installable et configurable, discret et l'\emph{overhead}
sur le système doit être minimisé. Il ne doit donc pas reposer
sur un autre système plus lourd comme le partage de fichiers.
Dans un premier temps, il ne doit gérer que le texte mais il doit être
extensible. Ceci afin de permettre facilement l'ajout du support pour
d'autres formats de données comme les images.
Une autre contrainte importante à prendre en considération
est l'aspect réseau. Celui-ci introduit de potentiels problèmes de sécurité.
Il faudra donc trouver un moyen de réduire ceux-ci \emph{e.g.} en chiffrant
la connexion réseau et en obligeant l'utilisateur à s'authentifier
afin d'utiliser le logiciel. De même, le logiciel devant tourner sur
plusieurs systèmes d'exploitations différents dont Linux, il est
plus que préférable que le logiciel soit libre de droits et se base
sur des protocoles et technologies ouverts.

\chapter{Analyse des solutions existantes}
\renewcommand{\leftmark}{\thechapter.~~Analyse des solutions existantes}
Cette section détaille l'analyse des solutions existantes
\emph{i.e.} la présentation
et comparaison de ces différentes solutions. Celles-ci sont divisées en
deux catégories. La première reprend les solutions dites \emph{lourdes},
\emph{i.e.} qui reposent sur des logiciels permettant de faire
beaucoup plus que du copier/coller \emph{e.g.} les \emph{bureaux virtuels}.
La seconde reprend les solutions \emph{potentiellement légères} \emph{i.e.}
qui sont des logiciels qui \emph{a priori} conviendraient comme solution
au problème.

Tout d'abord seront étudiés un ensemble de logiciels et protocoles
donnés comme mots-clés par le directeur de projet au début de celui-ci.
Ceux-ci sont principalement liés à des technologies lourdes et seront
majoritairement ceux entrant dans la partie des solutions lourdes. Les
autres solutions lourdes seront surtout des solutions plus conventionnelles
comme l'échange d'e-mails ou le partage de fichiers.

\section{Solutions lourdes}
Les solutions étudiées dans cette section seront d'abord l'échange
d'e-mails, l'utilisation d'un serveur \emph{FTP} (File Transfer Protocol)
et le partage de fichiers.
Les autres solutions étudiées seront \emph{Remote Desktop Service},
\emph{VNC}, \emph{Citrix XenApp} et
la technologie \emph{NX}. La comparaison de ces solutions est résumée
dans la table \ref{tbl:comp_lourd}.

\subsection{Échange d'e-mails}
Une première solution qui vient rapidement à l'esprit est l'utilisation
du courrier électronique. Il est en effet facile d'échanger du texte ou
un fichier quelconque entre deux machines en utilisation celui-ci.
Par exemple il suffit simplement d'envoyer à sa propre adresse
le texte à copier/coller ou bien de joindre le fichier que l'on désire
transférer. Cette solution à l'avantage d'être utilisable sur n'importe
quelle machine (et n'importe quel système d'exploitation) sans l'installation
d'un logiciel particulier autre qu'un client mail.

Cependant ceci reste très fastidieux. Outre l'utilisation d'un outil
qui n'est pas prévu pour faire du copier/coller, ceci nécessite
l'installation d'un serveur mail sur le réseau local, ou bien un accès à
Internet. Cet accès à Internet pose d'ailleurs un problème important
à l'heure actuelle où les connexions fournies aux particuliers ne
proposent pas une vitesse d'envoi très élevée. Un réseau intranet câblé
en \emph{Fast Ethernet} permettra généralement un débit symétrique d'au
moins 100 Mbits/s, alors qu'une connexion ADSL standard dépasse rarement
les quelques Mbits/s en upload. Les différences de latence entre l'internet
et l'intranet risquent aussi d'être fort importantes.

Ceci permet de montrer que, non seulement il n'est pas nécessaire
d'utiliser Internet pour échanger des données entre deux machines
qui sont normalement situées dans la même pièce, mais que cela peut
en plus apporter une perte de performances si l'on dispose d'une
connexion à faible débit\footnote{Ce qui est actuellement le cas en Belgique
ainsi que dans la plus grande partie du monde où la fibre optique et
les connexions symétriques sont peu répandues}.
L'utilisation d'Internet sera donc écartée par
la suite pour ces raisons et seule l'utilisation sur un réseau local
sera considérée.

\subsection{Utilisation d'un serveur FTP}
Un moyen de faire du copier/coller entre plusieurs plateformes
est l'utilisation d'un serveur FTP. Pour cela il faut qu'un
serveur soit installé sur le réseau local. De même chaque client
devra disposer d'un client FTP. Sur Unix il est aussi possible de se passer
d'un serveur FTP et d'utiliser un serveur \emph{SSH} (Secure Shell Client),
le protocole \emph{SFTP} (SSH File Transfer Protocol) permettant
d'utiliser un serveur SSH à la manière d'un serveur FTP sécurisé.

La solution du serveur FTP a donc comme contrainte l'installation du logiciel
serveur. Celle-ci peut être en partie résolue sous Unix grâce à SSH.
En effet il peut être assez fréquent d'avoir un serveur SSH installé
en local pour l'utilisateur Unix utilisant plusieurs machines simultanément.
Cependant le fait de créer un fichier pour copier/coller du texte engendre
une lourdeur non désirée. Cette dernière remarque s'applique aussi pour
le partage de fichiers.

\subsection{Partage de fichiers}
Le partage de fichier permet d'accéder à un répertoire se trouvant sur
une machine A à partir d'une machine B se situant sur le même réseau.
Celui-ci peut être mis en oeuvre de plusieurs manières. Sur Windows
il est implémenté nativement via le protocole SMB. Sous Unix le protocole
\emph{NFS} (Network File System) est sans doute le plus répandu.
Une alternative peut aussi être l'utilisation de SSH et \emph{SSHFs}
(SSH Filesystem qui permet de monter un dossier sur une machine distante
comme si c'était un périphérique et qui repose sur SFTP). Le partage de
fichiers de Windows n'étant pas compatible sous Unix et \emph{vice versa},
l'utilisation de \emph{Samba} \cite{samba} sera obligatoire s'il est nécessaire
de supporter ces deux mondes. Cette solution présente aussi les mêmes
désavantages que l'utilisation de FTP.

\subsection{Remote Desktop Services}
Remote Desktop Services, autrefois \emph{Terminal Services}
est un composant de Microsoft Windows permettant l'utilisation
d'un ordinateur à distance tournant sous Windows\cite{wiki:rds}.
Ce système est basé sur un modèle client-serveur.
Le serveur est appelé Terminal Server et est inclus dans Windows.
Il est à noter que seule la version serveur
de Windows permet une configuration avancée du programme serveur.
Le protocole utilisé est appelé \emph{RDP} (Remote Dekstop Protocol) et
peut être transporté dans un tunnel \emph{TLS} (Transport Layer Security,
anciennement \emph{SSL}, Secure Sockets Layer) afin d'améliorer la sécurité
du protocole. L'utilisation de ce protocole est possible sous les
systèmes d'exploitation basés sur Unix grâce à l'implémentation
libre \emph{rdekstop} \cite{rdesktop} du client. Le protocole et le serveur
supportent le partage du presse-papier, cependant il est évident que cette
solution est inadéquate pour être utilisée comme copier/coller
multi-plateformes.

\subsection{VNC}
VNC (Virtual Network Computing) \cite{wiki:vnc} est un système logiciel
permettant d'utiliser
un ordinateur à distance. Il a comme avantage sur le protocole de Microsoft
d'être libre de droits et d'être indépendant du système d'exploitation.
Bien que non sécurisé par défaut, il existe différents moyens de le sécuriser
\emph{e.g.} via une connexion SSH ou \emph{VPN} (réseau privé virtuel).
Cette solution souffre des mêmes problèmes que Remote Desktop Services,
c'est-à-dire qu'il permet de faire bien plus que du copier/coller et
allourdi le système.

\subsection{Citrix XenApp}
Citrix XenApp \cite{wiki:xenapp} est un ensemble de produits permettant de
virtualiser des applications sur différentes machines. Il distribue
des services tournant généralement sur un ou plusieurs serveurs à des
clients dits légers, \emph{i.e.} qui n'ont pas besoin d'avoir une grande
quantité de ressources matérielles disponibles pour exécuter
les applications, vu que celles-ci sont exécutées sur un serveur central.
Contrairement à VNC qui ne distribue que ce qui est
affiché, XenApp fonctionne de manière semblable à \emph{X11}
\footnote{X Window, X11 ou X est le système
graphique standard de Unix fonctionnant sous forme de serveur et où
chaque application graphique est un client}. Cependant ceci ne l'empêche
pas de souffrir de problèmes déjà évoqués, tels un overhead important lorsque
l'on veut faire du copier/coller et l'utilisation de license propriétaire.

\subsection{Technologie NX}
NX \cite{wiki:nx} est un protocole d'accès distant à X11 reposant sur un
modèle client-serveur et utilisant SSH pour la sécurité. L'implémentation
de base \emph{NoMachine NX} est propriétaire mais une implémentation libre
\emph{FreeNX} \cite{freenx} existe. Tout comme les solutions précédentes, NX
permet bien plus que le copier/coller et souffre donc des mêmes problèmes.

\subsection{ClusterSSH}

\subsection{Comparaison}
En résumé, toutes ces solutions lourdes présentées reposent
sur un modèle client-serveur et sont conçues, soit pour fournir un service
qui n'est pas prévu pour être utilisé afin de réaliser du copier/coller,
soit pour être utilisées comme bureau virtuel distant. Certaines sont
plus simples que d'autres à mettre en oeuvre; d'autres sont plus sécurisées,
tandis que d'autres sont propriétaires ou visent un système d'exploitation
particulier. Mais elles sont toutes inadaptées pour effectuer
un copier/coller multi-plateformes simple. Leurs caractéristiques sont résumées
dans la table \ref{tbl:comp_lourd}.

\begin{sidewaystable}[!h]
  \centering
  \begin{tabular}{|l|l|l|m{7em}|m{7em}|m{7em}|}
    \hline
    Solution & Service rendu & Architecture & Sécurité & In\-dé\-pen\-dance
    de la plateforme & Ouverture de la solution \\
    \hline
    \hline
    E-mails & E-mails & Client-serveur & Dépend du protocole & Oui &
    Protocoles ouverts \\
    \hline
    FTP & Transfert de fichiers & Client-serveur & SSH grâce à SFTP & Oui
    sauf SFTP & Protocoles ouverts \\
    \hline
    Partage de fichiers & Partage de fichiers & Client-serveur & SSH sous
    Unix & Oui grâce à Samba & Libre sous Unix, fermé sous Windows \\
    \hline
    RDS & Bureau distant & Client-serveur & Tunnel TLS possible & Windows
    mais clients Unix existants & Protocole propriétaire\\
    \hline
    VNC & Bureau distant & Client-serveur & Possibilité d'utiliser SSH ou un
    VPN & Oui & Logiciel libre \\
    \hline
    XenApp & Bureau distant & Client-serveur & Possibilité d'utiliser HTTPS &
    MS Windows Server, HP-UX, Solaris, AIX & Logiciel propriétaire \\
    \hline
    NX & Bureau distant & Client-serveur & Oui & Vise Unix &
    Im\-plé\-men\-ta\-tion libre FreeNX \\
    \hline
  \end{tabular}
  \caption{\label{tbl:comp_lourd} Comparaison des solutions lourdes}
\end{sidewaystable}

\section{Solutions \emph{a priori} légères}
Les solutions présentées dans cette section sont pour la majorité des
solutions trouvées en faisant des recherches sur Google sur base
de mots clés français et anglais. Celles-ci ont permis de trouver des
logiciels qui conviendraient au premier abord en fournissant un moyen
simple de faire du copier/coller en réseau. Les logiciels présentés seront
\emph{ClipboardMultiSharer}, \emph{Clipboard Share},
\emph{The Network Clipboard} et \emph{Remote Clip}.
La comparaison de ces solutions est résumée dans la table \ref{tbl:comp_leger}.

\subsection{Cl1p}

\subsection{ClipboardMultiSharer}
ClipboardMultiSharer \cite{clipmsharer} est un logiciel écrit en \emph{Java}
et en \emph{C\#} qui permet de faire du copier/coller entre plusieurs
ordinateurs. Celui-ci supporte le copier/coller de texte et d'image et repose
sur l'utilisation d'un fichier partagé en réseau. Ceci en fait donc en réalité
une solution lourde vu qu'il requiert l'utilisation du partage de fichiers.
De plus, bien que le programme soit écrit en Java, la portabilité du programme
dépend du type de partage de fichiers utilisé. Il sera donc sans doute
nécessaire d'utiliser Samba s'il est nécessaire de travailler entre Unix
et Windows.

\subsection{Clipboard Share}
Clipboard Share \cite{clipshare} est un autre logiciel qui permet de faire
du CCMP. La connexion est chiffrée et il est possible de recevoir
du contenu en provenance d'un envoyeur de confiance. Il est écrit en C\#,
requiert \emph{Microsoft .NET 3.5} ainsi que \emph{PNRP} (\emph{Peer Name
Resolution Protocol}), un protocole \emph{P2P} (Peer-to-Peer, pair à pair en
français, le principe de ce type de réseaux est décrit dans la section
\ref{sec:p2p} à la page \pageref{sec:p2p}) propriétaire de Microsoft,
ce qui signifie que le logiciel ne tourne que sur Windows XP SP2 ou plus
récent \cite{wiki:pnrp}. Bien qu'il propose des fonctionnalités intéressantes
et réponde au critère de logiciel léger, il ne convient pas car il ne vise
que Windows et repose sur des technologies propriétaires.

\subsection{The Network Clipboard}
The Network Clipboard \cite{netclip} est lui écrit en \emph{C++} et fonctionne
aussi bien sous Windows que sous Linux. Le protocole mis en place permet
d'utiliser l'application en P2P grâce au \emph{broadcast} \emph{IP}.
Cependant il possède plusieurs défauts qui l'empêchent d'être un bon candidat.
Premièrement le contenu n'est pas chiffré sur le réseau et aucun moyen
d'authentification ne semble être mis en oeuvre pour sécuriser le système.
Ensuite il ne semble pas certain que le programme tourne sur tous les Unix,
\emph{e.g.} il n'est fait mention nulle part d'une compatibilité assurée
avec \emph{MacOS}. Enfin, il faut tout de même noter que The Network
Clipboard utilise la version 3 de \emph{Qt}\footnote{framework C++ utilisé
entre autre dans le projet \emph{KDE}} et qui, avant la version 4, n'était
libre que sous Linux \cite{wiki:qt}.
Le programme repose donc sur une librairie propriétaire sous les autres
systèmes d'exploitation.

\subsection{Remote Clip}
Remote Clip \cite{remoteclip} est à la base un outil pour synchroniser
du contenu entre Windows et un \emph{PDA} \emph{Palm}. Celui-ci existe
aussi en version Java et lui permet ainsi d'être portable. Il repose sur une
architecture P2P dont le fonctionnement est expliqué dans un article de Robert
C. Miller et Brad A. Myers \cite{Miller99syncclips}. Les connexions sont
également chiffrées via TLS à partir de la version 1.4 de Java et
l'ajout d'un pair dans un groupe de partage du presse-papier nécessite
l'accord explicite de la machine gérant celui-ci. Il permet de gérer aussi
bien le texte que les fichiers et est distribué sous licence libre.
Remote Clip semble donc être le logiciel répondant à toutes les
exigences requises mais contrairement aux solutions citées plus haut,
le projet ne semble plus actif. En effet la dernière version du logiciel
est datée de juillet 2002. Cela signifie que de potentielles failles
de sécurité ne seront pas corrigées et que le logiciel ne sera pas amélioré.

\subsection{Comparaison}
En résumé toutes les solutions présentées ont chacune des qualités et des
défauts. ClipboardMultiSharer repose sur du partage de fichiers (et donc un
modèle client-serveur). Clipboard Share repose sur des technologies fermées
et ne tourne que sous Windows. The Network Clipboard n'est pas sécurisé.
Seul Remote Clips répond réellement aux exigences du projet même si celui-ci
est quelque peu vieillissant. Pour cette raison c'est les concepts de ce
dernier qui seront principalement utilisés pour développer le projet.
Les caractéristiques de l'ensemble des logiciels sont résumées dans la
table \ref{tbl:comp_leger}.

\begin{sidewaystable}[!h]
  \centering
  \begin{tabular}{|l|l|l|m{7em}|m{7em}|m{7em}|}
    \hline
    Solution & Service rendu & Architecture & Sécurité & In\-dé\-pen\-dance
    de la plateforme & Ouverture de la solution \\
    \hline
    \hline
    Clipboard\-MultiSharer & CCMP & Client-serveur & Partage de fichiers & Java
    + partage de fichiers & Logiciel libre\\
    \hline
    Clipboard Share & CCMP & P2P & Connexion cryptée + envoyeur de confiance &
    Windows (.NET 3.5 + PNRP) & Logiciel libre mais technologies MS \\
    \hline
    The Network Clipboard & CCMP & P2P & aucune & Linux + Windows &
    Logiciel libre \\
    \hline
    Remote Clip & CCMP & P2P & TLS + validation de connexion d'un pair &
    Java & Logiciel libre \\
    \hline
    \hline
    Solution & Service rendu & Architecture & Sécurité & In\-dé\-pen\-dance
    de la plateforme & Ouverture de la solution \\
    \hline
  \end{tabular}
  \caption{\label{tbl:comp_leger} Comparaison des solutions légères}
\end{sidewaystable}

\chapter{Solution proposée}
\renewcommand{\leftmark}{\thechapter.~~Solution proposée}
Après analyse des différentes solutions existantespour faire du CCMP dans
le chapitre précédent, la solution jugée la plus efficace
est Remote Clip, celle-ci proposant une architecture P2P adéquate et un
niveau de sécurité correct. Pour ces raisons, certaines idées mises en avant
par Remote Clip sont réutilisées ici. De même le principe de messages
broadcast utilisé dans le but d'auto-configurer le réseau, comme fait dans The
Network Clipboard, est aussi repris.

\section{Principes d'une architecture P2P}\label{sec:p2p}
Avant de décrire l'architecture qui utilisée, il est préférable
de rappeler quels sont les principes de base d'une architecture P2P.
Habituellement, une architecture client-serveur est utilisée lorsqu'il
faut fournir un service à un ensemble de machines connectées en réseau,
c'est-à-dire que chaque client va se connecter à un (ou parfois plusieurs)
serveur central qui s'occupera de fournir le service désiré. Par opposition à
ce mode de fonctionnement centralisé, un réseau organisé en pair-à-pair
permet de se passer de serveur central, chaque client jouant à la fois le rôle
de client et de serveur. La figure \ref{fig:p2p} illustre la différence
de topologie réseau existante entre ces deux types d'architectures réseau.
De manière plus précise les systèmes P2P sont définis dans \cite{AS04} comme
étant:
\begin{quote}
  des systèmes distribués constitués de noeuds interconnectés, capables de
  s'auto-organiser dans des topologies de réseaux avec comme but le partage
  de ressources, telles que le contenu, les cycles CPU, le stockage,
  la bande passante tout en ayant la capacité de s'adapter aux erreurs et
  de s'accommoder de populations de noeuds transitoires; tout en maintenant
  une connectivité et des performances acceptables sans requérir
  l'intermédiaire ou le support d'un serveur ou d'une autorité
  centralisée globalement.
\end{quote}

Les difficultés potentielles à mettre en évidence dans ce genre de systèmes
sont la gestion des va-et-vient de clients et la tolérance aux erreurs
sans l'aide de serveur central, et tout en minimisant la charge sur le réseau,
due à cette gestion. Il faut cependant noter que l'importance de la tolérance
aux erreurs est tout de même assez minime dans le contexte du copier/coller.
Les données stockées sont en général destinées à être utilisées à court terme
et non pas à être stockées de manière persistante. De même le nombre de pairs
est normalement peu élevé, et donc le nombre de va-et-vient n'est pas aussi
important que sur des systèmes à grande échelle.

Une remarque supplémentaire à faire est qu'une différence est souvent faite
entre réseaux P2P structurés et non structurés \cite{AS04, Lua05asurvey}.
Les premiers ont une topologie contrôlée de manière précise permettant
de placer chaque donnée à un endroit précis et d'effectuer des recherches
efficaces. Le réseau ici étant censé être de taille relativement petite
et le contenu changeant très rapidement, il n'est pas nécessaire de structurer
le réseau. En effet, le fait de structurer le réseau, en utilisant par exemple
une table de hashage distribuée, a comme intérêt d'avoir une complexité
algorithmique sous-linéaire en fonction du nombre de pairs présent dans
le réseau. Ici la taille du réseau étant fortement réduite, cette
complexité a peu d'impact sur les performances du logiciel. Le fait de ne
pas utiliser une table de hashage distribuée permet donc d'avoir un protocole
plus simple sans perte de performances.

\begin{figure}[!h]
  \centering
  % \input{fig_p2p.tex}
  \includegraphics[width=\textwidth]{fig_p2p}
  \caption{Exemple de différence entre une topologie client-serveur et P2P.}
  \label{fig:p2p}
\end{figure}

\section{Architecture logicielle}
Afin de suivre le principe \emph{KISS} (Keep It Simple Stupid, \emph{cf.}
Annexe \ref{ann:kiss}) et la philosophie Unix, le projet est divisé en
plusieurs programmes, chacun d'entre eux s'occupant d'une tâche particulière.
Une autre raison justifiant une telle conception est le fait qu'elle introduit
la possibilité de développer le projet de manière incrémentale, en se
concentrant d'abord sur l'aspect réseau et ensuite sur l'implémentation
propre à un environnement particulier.

\subsection{Client P2P}
Premièrement un logiciel s'occupant uniquement de la gestion du presse-papier
sur le réseau est nécessaire. Celui-ci a comme rôle de se connecter aux
autres pairs et de s'organiser avec ceux-ci quand cela est nécessaire.
Sa seule fonctionnalité est en fait de gérer la partie réseau du système,
toute interaction locale (\emph{i.e.} avec l'utilisateur) se fait par
l'intermédiaire d'autres logiciels qui seront définis par la suite.

Il faut définir de manière plus précise comment l'ensemble des clients P2P
vont s'organiser afin de s'informer pour savoir quel client \emph{détient} le
presse-papier, comment un pair entre dans le réseau et comment détecter
qu'un pair a quitté celui-ci (volontairement ou pas).

\subsubsection{Gestion du presse-papier en P2P}
Il y a principalement deux possibilités pour gérer le presse-papier
sur le réseau. Lorsqu'un copier est effectué en local, le client P2P
doit prévenir les autres pairs qu'il détient le presse-papier.
La première option est d'envoyer en même temps la donnée copiée à tous les
pairs. La deuxième est de n'envoyer cette donnée que lorsque qu'un coller
est effectué chez un pair, celui-ci peut alors demander la donnée au pair
détenant le presse-papier.
La première manière de faire a les avantages et inconvénients suivants:
\begin{itemize}
\item Avantages:
  \begin{itemize}
  \item Possibilité de garder une copie du presse-papier sur chaque pair.
  \item Tolérance aux erreurs dans le cas où le pair détenant le presse-papier
    aurait quitté le réseau.
  \item Absence de communications réseaux en cas de multiples coller.
  \end{itemize}
\item Inconvénients:
  \begin{itemize}
  \item Consommation de bande passante accrue dans le cas où des copiers
    sont effectués fréquemment car il faudra envoyer la même donnée à chacun
    des pairs qui ne feront peut être pas forcément de coller. De plus la
    consommation de bande passante dépend du nombre de pair présent sur le
    réseau (même si celui-ci reste faible).
  \end{itemize}
\end{itemize}
La seconde a les avantages et inconvénients suivants:
\begin{itemize}
\item Avantages:
  \begin{itemize}
  \item Il n'est pas nécessaire de notifier l'ensemble des pairs à chaque
    copier: si un pair effectue plusieurs copies d'affilée, il ne faudrait
    notifier le réseau qu'une seule fois et sans envoyer la moindre donnée.
  \item En général le coller ne sera effectué que par un seul pair,
    il n'a donc qu'à demander lui même au pair détenant le presse-papier
    de la lui envoyer.
  \end{itemize}
\item Inconvénients:
  \begin{itemize}
  \item Faible tolérance aux erreurs: impossibilité de récupérer le
    pres\-se-pa\-pier d'un pair ayant quitté le réseau.
  \item Si plusieurs collers ont lieu les uns à la suite des autres, il faudra
    demander plusieurs fois la même donnée au pair détenant le presse-papier.\\
  \end{itemize}
\end{itemize}

\subsubsection{X Window}
Avant de choisir entre une de ces deux solutions, il est important de
savoir ce qu'est X Window et comment ce logiciel fonctionne\cite{nye1992xlib}.
X est un standard pour les systèmes d'exploitations basés sur Unix permettant
d'afficher des éléments graphiques sur un écran plutôt qu'un terminal.
Il est principalement responsable de la couche la plus basse des systèmes
de fenêtrages de la majorité des systèmes Unix. Son fonctionnement est basé
sur une architecture client-serveur où X Window est le logiciel serveur
et chaque application graphique est un client. Lorsqu'un copier est réalisé au
sein d'une application, le client (\emph{i.e.} la fenêtre) notifie le serveur X
qu'un copier a eu lieu. Le serveur retient alors que la dernière application
à avoir copié du texte est le client en question.

Lorsqu'une deuxième
application effectue un coller, elle demande d'abord au serveur quel client
a fait le dernier coller. Il peut ensuite demander à ce client de délivrer
le contenu de son presse-papier. Il faut noter que si le client responsable
du presse-papier a disparu, \emph{i.e.} si la fenêtre a été fermée, et qu'aucun
copier n'a été effectué depuis la fermeture du client, alors il est
impossible de récupérer la donnée et le presse-papier est considéré comme vide.
\footnote{Il faut cependant noter que si un logiciel d'historique de
presse-papier est utilisé, c'est le gestionnaire de presse-papier qui est
responsable de la donnée copiée. Il est donc normal si celle-ci est toujours
disponible après avoir fermé la fenêtre dans laquelle la donnée a été
initialement copiée.}
En fin de compte, il est clair que la description du fonctionnement de X Window
est la deuxième solution décrite précédemment.

\subsubsection{Choix d'une solution pour la gestion et
synchronisation du presse-papier}
Il faut donc faire un choix entre une de ces deux solutions. Un des critères
définit au début de ce travail étant de cibler Unix, il semble idéal de se
baser sur X Window pour implémenter une nouvelle solution. De même, Remote
Clip, ayant été décrit comme la meilleure solution rencontrée, fonctionne
de la même manière. Cependant en fin compte, c'est la première solution
envoyant le presse-papier lors de la copie qui est choisie.
Celle-ci est plus intuitive et a surtout l'avantage d'être
tolérante aux erreurs. Et même si Remote Clip a été identifiée comme
solution idéale, un intérêt de réimplémenter une nouvelle solution est
d'améliorer ses fonctionnalités, en améliorant ici la tolérance aux erreurs.
Un dernier avantage qui est discuté dans le chapitre \ref{chap:implem}
est le fait qu'envoyer la donnée à chaque pair à chaque copier facilite
la gestion d'historique synchornisé du presse-papier sans avoir à ajouter
cette fonctionnalité directement dans le logiciel nouvellement produit.

\subsubsection{Rejoindre le réseau}
Outre un moyen d'authentification, il est nécessaire de définir comment
un pair peut rejoindre le réseau afin de partager son presse-papier.
Une première solution consiste, pour un pair voulant
rejoindre le réseau, à connaître l'adresse IP et l'identifiant d'au moins un
autre pair déjà présent dans le réseau et connaissant les autres pairs présents
dans ce réseau. Il suffit alors de contacter cet autre pair et de synchroniser
la  liste des pairs de l'ensemble du réseau. Une autre solution envisageable
est d'utiliser un mécanisme de broadcast permettant de flooder
\footnote{Bien que le fait d'innonder le réseau de messages ait un aspect
négatif, la quantité de bande passante reste cependant faible, les messages
envoyés étant relativement courts.} le réseau pour
rechercher les pairs désirant rejoindre le réseau. Cette solution sera celle
utilisée, celle-ci permettant de minimiser le nombres de paramètres que devra
configurer l'utilisateur.

Pour cela, il faut que chaque pair envoie à interval régulier un message
en broadcast, sur le réseau local, permettant d'identifier le groupe de pairs
(via un nom choisit par l'utilisateur)
auquel il appartient ou veut appartenir. Tout pair appartenant au groupe
(\emph{i.e.} dont le groupe possède le même nom) et
recevant un message doit alors contacter le pair ayant envoyé le message en
ouvrant une connexion TCP. L'intérêt de cette connexion TCP est de s'assurer
que le presse-papier est reçu dans son intégrité par chaque pair (les
messages broadcast étant envoyés via des paquets UDP). De plus,
ceci permet à chaque pair de s'identifier afin de s'assurer que seuls
les membres du groupe (\emph{i.e.} les machines de l'utilisateur présentes
sur le réseau local) ont accès au presse-papier. Pour cela un
mécanisme d'authentification est décrit dans la suite de ce chapitre.

Il faut cependant faire attention à un
problème: le nombre de connexions TCP ouvertes peut devenir une charge
importante pour le réseau. En effet pour un groupe de $n$ pair, $n-1$
connexions sera ouverte pour chaque pair. Le nombre de connexions serait donc
en $O(n^2)$. Cependant le réseau P2P étant limité à quelques pairs dans le
cas du CCMP, ceci n'est pas handicapant.
De même afin d'ouvrir plusieurs fois une connexion TCP avec un pair déjà
connu, il faut gérer une table des pairs dans laquelle sera ajouté un
pair identifié. Les figures \ref{fig:broadcast} et \ref{fig:connections}
illustrent par des diagrammes d'états comment un pair se comporte pour
contacter les autres pairs du réseaux.

\begin{figure}[!h]
  \centering
  \includegraphics{broadcast}
  \caption{Diagramme d'états modélisant l'envoi de messages en broadcast
    par le client P2P.}
  \label{fig:broadcast}
\end{figure}

\begin{figure}[!h]
  \centering
  \includegraphics[width=\textwidth]{connections}
  \caption{Diagramme d'états modélisant le comportement du client P2P lors
    de la réception d'un message broadcast et lors de l'acceptation d'une
    connexion TCP.}
  \label{fig:connections}
\end{figure}

\subsubsection{Quitter le réseau}
Lorsqu'un pair quitte le réseau, il doit notifier l'ensemble des autres
pairs de son départ. Une fois ceci fait, il peut fermer chaque connexion
ouverte avec chacun des pairs. En revanche il se peut qu'un pair quitte
le réseau sans pouvoir notifier les pairs de son départ, \emph{e.g.}
à cause d'une déconnexion du lien physique. Pour cela un mécanisme
de \emph{keep alive} doit être utilisé. Il faut que chaque pair envoie
de manière régulière un message sur chaque connexion ouverte, même en cas
d'inactivité de la part de l'utilisateur. Lorsqu'un pair n'envoie plus
de messages keep alive pendant un certain laps de temps, il faut que les
autres pairs ferment la connexion TCP avec ce pair et le considèrent comme
ayant quitté le réseau. Si la connexion physique est rétablie, le pair
\emph{disparu} peut alors rejoindre à nouveau le réseau grâce au mécanisme
de broadcast.

\subsubsection{Authentification des pairs}
Afin d'assurer que le contenu du presse papier reste confidentiel à son
utilisateur et qu'aucune personne présente sur le réseau ne soit capable de
modifier son contenu, un système d'authentification doit être mis en place.
Celui-ci permet de s'assurer qu'aucun utilisateur non désiré puisse accéder au
presse-papier.

Ce système d'authentification est basé sur le protocole de
Need\-ham-Schroe\-der
\cite{1978Needham} permettant l'échange sécurisé de clés de chiffrement
en utilisant une autorité de confiance. Ici cependant l'autorité de confiance
est l'utilisateur ayant au préalable configuré le logiciel.
Pour cela lors de l'ouverture d'une connexion TCP, chaque
pair envoie un message de type JOIN avec son nom de pair et un
\emph{nonce} chiffré. C'est un nombre généré de manière pseudo-aléatoire que
chaque pair devra renvoyer incrémenté de un. Ceci permet d'avoir un nombre
utilisé une seule fois et d'empêcher un utilisateur étranger
de rejoindre le réseau de pairs en réutilisant un message qui aurait déjà
été utilisé. Le comportement d'un pair ayant été contacté par un autre pair
est illustré par un diagramme d'états dans la figure \ref{fig:peerhandler}.
Le comportement pour le pair ayant initié le contact est le même.

\begin{figure}[!h]
  \centering
  \includegraphics[width=0.9\textwidth]{peerhandler}
  \caption{Diagramme d'états illustrant l'authentification des pairs lors
    de l'ouverture d'une session TCP. Le noeud INIT représente l'ouverture
    de la session et les autres noeuds représentent différents états. Les
    transitions sont déclenchées si la condition entre crochets et les actions
    présentes ensuites sont exécutées. Ce diagramme définit aussi bien le
    comportement du pair ayant initié la connexion TCP que celui l'ayant
    accepté.}
  \label{fig:peerhandler}
\end{figure}

\subsubsection{Envoi du presse-papier}
Le presse-papier est envoyé lorsqu'un client local notifie le client
P2P qu'un copier a été effectué.
Afin d'éviter que les keep alive ne soient plus transmis lors
de l'envoi du presse-papier contenant une donnée particulièrement
longue, la possibilité de fragmenter le presse-papier est permise dans
le protocole mis en place (\emph{cf.} page \pageref{msg:data_p2p}).

\subsection{Client local}
Le client P2P est le \emph{front-end} avec le réseau, il est chargé
de communiquer avec les pairs du réseau afin de les découvrir et savoir
lequel détient le presse-papier. Le front-end avec l'utilisateur
est en revanche le client local. Celui-ci tourne sur le même ordinateur
que le client P2P, se charge d'envoyer le contenu du presse-papier
lors d'une copie de l'utilisateur, et lui demande le contenu du presse-papier
lorsque que cet utilisateur effectue un coller.
De même le client P2P envoie le contenu du presse-papier au client local si
ce dernier a envoyé une requête pour l'obtenir ou si un autre
client local a changé le contenu du presse-papier.

Cependant, contrairement au client P2P, il peut y avoir plusieurs clients
locaux tournant sur le même ordinateur. Chacun d'entre eux étant destiné
à un environnement précis. Dans ce projet, seront développés deux types
de clients différents:
\begin{itemize}
\item Un client tournant en tâche de fond (comme \emph{daemon}) et
  synchronisant le contenu du presse-papier de X Window.
\item Un client gérant le copier/coller dans un terminal. L'intérêt de ce
  client est d'envoyer des résultats de commandes Unix directement via
  un pipe. Ceci étant plus intuitif que la copie via le système de fenêtrage
  lorsque l'on travaille en console.
  Il est composé de deux logiciels:
  \begin{itemize}
  \item Une commande permettant de copier une donnée à partir de l'entrée
    standard du shell.
  \item Une commande permettant de coller une donnée sur la sortie standard
    du shell.
  \end{itemize}
\end{itemize}

\section{Protocoles}
Trois protocoles sont à définir afin de faire fonctionner l'application
correctement. Le premier est celui permettant de découvrir des pairs grâce au
broadcast. Le second est celui utilisé sur le réseau par les clients
P2P afin de communiquer entre eux. Le dernier est utilisé entre le client
P2P et les clients locaux afin de se notifier mutuellement de changements
dans le presse papier. De même ils doivent prendre en compte
l'aspect sécurité, en proposant un moyen d'identification.
Ces trois protocoles seront appelés respectivement protocole broadcast,
protocole P2P et protocole local.

Les protocoles sont caractérisés par un ensemble de types de messages.
Ceux-ci sont décrits en utilisant la syntaxe suivant:
\begin{verbatim}
MSG <SP> <VAR1> <SP> <VAR2> ... <SP> <VARN> <CRLF>
\end{verbatim}
MSG définit le type du message. Celui-ci contient $N$ va\-ria\-bles
<VAR1>, <VAR2>, $\ldots$, <VARN>. Chacune est séparée par un espace (<SP>)
et le message se termine par un retour chariot (<CRLF>). Chaque variable est
ensuite décrite et le type de messages pouvant être reçu comme réponse est
décrit.

\subsection{Protocole broadcast}
\subsubsection*{JOIN}
Un message de type JOIN envoyé en broadcast permet d'informer les autres
pairs de l'existence de ce pair ci.
\begin{verbatim}
JOIN <SP> <NAME> <SP> <GROUP> <CRLF>
\end{verbatim}
\begin{description}
\item[Variable NAME:] identifiant du pair envoyant le message.
\item[Variable GROUP:] nom du groupe de pairs définit par l'utilisateur
  lors de la configuration du logiciel.
\item[Réponse:] si le pair n'est pas déjà présent dans la table des pairs,
  il faut ouvrir une connexion TCP avec celui-ci en utilisant le protocole P2P.
\end{description}

\subsection{Protocole P2P}
\subsubsection*{JOIN}
Un message de type JOIN sert à authentifier un pair.
\begin{verbatim}
JOIN <SP> <NAME> <SP> <CHALLENGE> <CRLF>
\end{verbatim}
\begin{description}
\item[Variable NAME:] identifiant du pair envoyant le message.
\item[Variable CHALLENGE:] un nombre généré aléatoirement utilisé comme
  challenge.
\item[Réponse:] un message ACCEPT où le challenge est incrémenté de un.
\end{description}

\hrulefill

\subsubsection*{ACCEPT}
Un message de type ACCEPT sert de réponse à un message JOIN.
\begin{verbatim}
ACCEPT <SP> <CHALLENGE> <CRLF>
\end{verbatim}
\begin{description}
\item[Variable CHALLENGE:] le challenge reçu dans le message JOIN
\item[Réponse:] Si le challenge ne correspond pas (\emph{i.e.} n'a pas été
  incrémenté de un et l'authentification a échoué), alors un KO est envoyé
  pour fermer la con\-ne\-xion.
\end{description}

\hrulefill

\subsubsection*{KO}
Ce type de message permet de signaler un refus ou une erreur et de fermer
la connexion.
\begin{verbatim}
KO <SP> <ERRNO> <CRLF>
\end{verbatim}
\begin{description}
\item[Variable ERRNO:] code d'erreur pouvant être:
  \begin{description}
  \item[0] fermeture de la connexion.
  \item[1] fermeture due à un timeout.
  \item[2] message ACCEPT non valide, échec de l'authentification.
  \item[3] type de données du message DATA inconnu. Pas de fermeture de la
    session, le contenu du presse-papier est simplement ignoré par le pair.
  \end{description}
\end{description}

\hrulefill

\subsubsection*{OK}
Ce type de message est envoyé à intervalle régulier afin de servir de
mécanisme de keep alive.
\begin{verbatim}
OK <CRLF>
\end{verbatim}

\hrulefill

\subsubsection*{DATA}
\label{msg:data_p2p}
Un message de type DATA permet d'envoyer le contenu du presse-papier.
Le contenu presse-papier peut être fragmenté et être envoyé dans plusieurs
messages DATA. Il n'est cependant pas
nécessaire de numéroter les messages dans le cas de données fragmentées,
TCP permettant de s'assurer que les messages sont reçus dans l'ordre
dans lequel ils ont été envoyés. De même l'implémentation du protocole doit
s'assurer qu'un pair n'envoie pas simultanément deux données à un même autre
pair.
\begin{verbatim}
DATA <SP> <TYPE> <SP> <MORE> <SP> <LENGTH> <SP>
  <CONTENT> <CRLF>
\end{verbatim}
\begin{description}
\item[Variable TYPE:] cette variable permet de préciser le type
  de donnée et ainsi d'étendre facilement le protocole afin de supporter
  d'autres types de données. Dans ce cas il faudra préciser comment ce type
  de données est encodé.
  \begin{description}
  \item[0] texte.
  \end{description}
\item[Variable MORE:] vaut 1 s'il y a encore des données fragmentées qui
  suivent, 0 sinon.
  Si la variable vaut 1, alors il faut bufferiser le contenu du message
  jusqu'à ce qu'à recevoir un message DATA avec cette variable égale à 0.
\item[Variable LENGTH:] longueur de la donnée en bytes.
\item[Variable CONTENT:] contenu de la donnée dont la longueur
  doit vérifier la variable LENGTH.
\end{description}

\subsection{Protocole local}
\subsubsection*{GET}
Ce type de message est envoyé par le client local pour demander au client P2P
d'envoyer le contenu du presse-papier.
\begin{verbatim}
GET <CRLF>
\end{verbatim}

\hrulefill

\subsubsection*{DATA}
Un message de type DATA permet d'envoyer le contenu du presse-papier. Un tel
message peut aussi bien être envoyé par le client P2P que par le client local.
\begin{verbatim}
DATA <SP> <TYPE> <SP> <LENGTH> <SP> <CONTENT> <CRLF>
\end{verbatim}
\begin{description}
\item[Variable TYPE:] cette variable permet de préciser le type
  de donnée et ainsi d'étendre facilement le protocole afin de supporter
  d'autres types de données. Dans ce cas il faudra préciser comment ce type
  de données est encodé.
  \begin{description}
  \item[0] texte.
  \end{description}
\item[Variable LENGTH:] longueur de la donnée en bytes.
\item[Variable CONTENT:] contenu de la donnée dont la longueur
  doit vérifier la variable LENGTH.
\end{description}

% \input{outils}
\chapter*{Conclusion}
\addcontentsline{toc}{chapter}{Conclusion}
\renewcommand{\leftmark}{CONCLUSION}
Finalement, après avoir examiné le problème, les solutiosn existantes
pour le résoudre se sont avérées pour la plupart inefficaces. Une solution
reprenant un grand nombre d'idées de The Network Clipboard et de Remote
Clip a été imaginée pour ensuite être implémentée. Cette solution
comprend un mécanisme de broadcast pour la découverte de nouveaux et
d'authentification permettant de s'assurer que le presse-papier ne pourra ni
être altéré, ni être lu par un utilisateur du réseau local étrangé au groupe
d'ordinateurs entre lesquels l'utilisateur souhaite synchroniser le
presse-papier.

Clipsync, le logiciel produit, a ainsi pu être publié comme logiciel libre sur
Launchpad dans le but de facilité sa distribution ainsi que son évolution.
Pour plannifier cette évolution, les fonctionnalités manquantes ont été mises
en avant.
Le passage à l'IPv6 étant un problème d'actualité urgent, l'effort
sera avant tout concentré sur son support dans Clipsync ainsi que sur la
découverte et la correction debugs. Ceci permettra de stabiliser le logiciel
avant l'ajout de nouvelles fonctionnalités.

Ensuite l'ajout de nouvelles fonctionnalités pour l'utilisateur telles que
l'implémentation du copier/coller d'images et la compression du presse-papier,
seront la seconde priorité des développements futurs du logiciel.
A plus long terme, l'utilisation de mécanismes d'auto-configuration
standardisé tels qu'Avahi sera sans doute utilisée pour remplacer le mécanisme
de broadcast.

%Le style bibliographique utilisé
% \bibliographystyle{latex8}
% \bibliographystyle{alpha}
% \bibliographystyle{alpha}

%Le fichier .bib uitilisé
% \bibliography{biblio}

\printbibliography

\newpage
\appendix
\addcontentsline{toc}{chapter}{Annexes}
\chapter{Codes sources utilisés}\label{ann:src}
\renewcommand{\leftmark}{ANNEXE \thechapter.~~Codes sources utilisés}
\label{annexe1}

Afin de comprendre comment fonctionnent certains logiciels étudiés dans
l'étude de faisabilité, certains codes sources disponibles en ligne
ont été utilisés. Ceux-ci étant disponibles sous licences libres (GPL et MIT),
ils sont non seulement disponibles en ligne mais peuvent aussi être
réutilisés librement.

Même si ces codes ne seront pas réutilisés pour des raisons évidentes
(utilisation d'un environnement de développement différent), certains
principes ont été compris grâce à ceux-ci. De même les principes de base
utilisé dans le projet s'inspirent largement de ces logiciels, c'est pour
cette raison qu'une annexe leur est consacrée.

\section*{ClipboardMultiSharer}
Le code source Java de l'application sous licence GPL
\footnote{\url{http://sourceforge.net/projects/clipboardmshare/develop}}
a été parcouru afin de comprendre comment fonctionnait
le logiciel, entre autres afin de voir comment il était possible d'accéder
au presse-papier en Java.
La révision consultée était la 16 datant du 10 juillet 2010.

\section*{Clipboard Share}
N'ayant pas de connaissances en C\# et en MS .NET, je n'ai ni lu, ni utilisé
le code de ce logiciel sous license MIT\footnote{\url{http://clipboardshare.codeplex.com/SourceControl/list/changesets}}.

\section*{The Network Clipboard}
Le code source sous licence GPL
\footnote{\url{http://sourceforge.net/projects/netclipboard/develop}}
a été consulté dans le but d'examiner le protocole utilisé. La révision
consultée était la 88.

\section*{Remote Clip}
Le code source sous licence GPL est disponible avec la dernière version
du logiciel
\footnote{\url{http://www.cs.cmu.edu/~rcm/RemoteClip/RemoteClip-3.1.zip}}.
Celui-ci a été consulté dans le but d'étudier le protocole réseau utilisé.

\section*{Gestionnaires de presse-papier}
Le code source de plusieurs gestionnaires de presse-papier ont permis d'aider
à la compréhension du fonctionnement du presse-papier via l'utilisation de
librairies annexes telles que GTK et Qt.
Ces logiciels sont Glipper 2.1\footnote{Version 2.1:
\url{https://launchpad.net/glipper}},
Klipper\footnote{Version présente dans KDE 4.6 (\url{http://www.kde.org})} et
parcellite \footnote{version 1.0.1: \url{http://parcellite.sourceforge.net}}.

\chapter{Principe KISS}\label{ann:kiss}
\renewcommand{\leftmark}{ANNEXE \thechapter.~~Principe KISS}
\label{annexe2}
Le principe \emph{KISS} (Keep It Simple, Stupid!) \cite{wiki:kiss} est un
principe prônant la simplicité, tout particulièrement dans le monde de la
conception.
Il est utilisé dans le développement logiciel dans le but d'exprimer le fait
que la conception doit être simple et éviter toute complexité inutile.

Celui-ci est à la base même de la philosophie Unix, en effet celle-ci prône
l'utilisation de petits utilitaires simples ayant chacun une fonctionnalité
bien précise.
Douglas McIlroy, inventeur du pipe UNIX, a dit\cite{quartercentury-unix}:
\begin{quote}
  This is the Unix philosophy: Write programs that do one thing and do
  it well. Write programs to work together. Write programs to handle
  text streams, because that is a universal interface.
\end{quote}
Le projet a donc été développé en utilisant une approche orientée
composant\cite{wiki:poc}, en fournissant de petits
logiciels communiquant au travers de sockets de manière textuelle.

Plusieurs auteurs
\cite{Brooks1995, Raymond2001} prônent cette philosophie sans pour autant
y faire référence explicitement. \emph{e.g.} Brooks explique par exemple
que le design est d'une importance capitale car il permet de comprendre
plus facilement le code source, Raymond dit qu'il faut concevoir les logiciels
de manière à ce qu'ils suivent la philosophie d'Unix. Il dit aussi que
la perfection en conception n'est pas atteinte lorsque l'on n'a plus rien à
ajouter mais plutôt lorsque l'on n'a plus rien à retirer.

Filip Hanik, un ingénieur logiciel membre de l'\emph{Apache Foundation} parle
aussi du principe KISS et donne des conseils pour appliquer ces principes en
Java (mais ceux-ci sont applicables dans tous les langages de programmation)
\cite{fhanikKISS}.


\end{document}
